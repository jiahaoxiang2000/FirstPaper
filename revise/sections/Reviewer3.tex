\section{Response to Reviewer \#3}
\subsection*{Overall Comments}
\begin{mdframed}
	\begin{quote}
		Reviewer \# 3 - This work implemented the CRAFT block cipher with two architectures proposed, the Serial architecture and the Unrolled architecture. Then, the efficiency of the implementations is tested on three FPGA platforms. The implementations proposed in this paper are interesting for the practical applications of CRAFT in IoT.
	\end{quote}
\end{mdframed}

\subsection{Response}

We sincerely appreciate the time and effort you've invested in reviewing our paper. Your valuable suggestions are greatly appreciated and will undoubtedly enhance the quality of our work. In the subsequent section, we will address each of your comments individually.

\noindent\rule{\linewidth}{2.0pt}

\subsection{Reviewer Comment}
\begin{mdframed}
	\begin{quote}
		Reviewer \# 3.1 - The reference format should be consistent with equations, such as, 'as shown in Equation 1' $->$ 'as shown in Equation (1)'.
	\end{quote}
\end{mdframed}

\subsection{Response}

Thank you for your suggestion. We have revised the manuscript to ensure that the reference format is consistent with the equations.


\noindent\rule{\linewidth}{2.0pt}

\subsection{Reviewer Comment}
\begin{mdframed}
	\begin{quote}
		Reviewer \# 3.2 - The description of IA is not necessary to occupy a single subsection (Subsection 3.3). It is better to move it to Preliminaries. Meanwhile, the comparison between the IA, SA, and UA is not sufficient.
	\end{quote}
\end{mdframed}

\subsection{Response}

todo for Response to Reviewer \# 3.2

1. remove subsection 3.3 IA ?

2. add the discussion of the comparison between the IA, SA, and UA on discussion section 

\noindent\rule{\linewidth}{2.0pt}

\subsection{Reviewer Comment}
\begin{mdframed}
	\begin{quote}
		Reviewer \# 3.3 - The definitions of SA and UA are not explicit, especially the implementation under UA. It is necessary to highlight their innovativeness in relation to previous architectures.
	\end{quote}
\end{mdframed}

\subsection{Response}

todo for Response to Reviewer \# 3.3

in the 3.1 and 3.2 header, we will add the definition of SA and UA, and highlight their innovativeness in relation to previous architectures.

\noindent\rule{\linewidth}{2.0pt}

\subsection{Reviewer Comment}
\begin{mdframed}
	\begin{quote}
		Reviewer \# 3.4 - Can CRAFT be implemented under the Iterative architecture? If so, do implementations under UA or SA still have advantages over IA? On the other hand, can UA or SA be applied to other ciphers, such as PRESENT, and bring efficiency improvements?
	\end{quote}
\end{mdframed}

\subsection{Response}

todo for Response to Reviewer \# 3.4, it want a fair compare, give it a reason.

logic answer:

The CRAFT can be implemented under the Iterative architecture, but the UA and SA still have advantages over IA. The UA and SA can also be applied to other ciphers, such as PRESENT, to bring efficiency improvements. However, since different ciphers have different linear and non-linear operations, enabling the new components to work together poses a challenge under a common strategy.

\noindent\rule{\linewidth}{2.0pt}

\subsection{Reviewer Comment}
\begin{mdframed}
	\begin{quote}
		Reviewer \# 3.5 - It is better to combine the Section 4 with Section 5 in my opinion.
	\end{quote}
\end{mdframed}

\subsection{Response}

Thank you for your suggestion. To enhance the readability of the manuscript, we have merged Sections 4 and 5 into a new Section 4 titled "Experiment Results". The previous Section 5 has been transformed into Subsection 4.5.

\noindent\rule{\linewidth}{2.0pt}

\subsection{Reviewer Comment}
\begin{mdframed}
	\begin{quote}
		Reviewer \# 3.6 - The data in Figure 9-13 is also demonstrated in Table 5-7, one approach is enough to compare the results.
	\end{quote}
\end{mdframed}

\subsection{Response}

todo for Response to Reviewer \# 3.6

add a horizontal comparison Table, not use the Figure 9-13.

\noindent\rule{\linewidth}{6.0pt}
\section{Response to Reviewer \#3}
\subsection*{Overall Comments}
\begin{mdframed}
	\begin{quote}
		Reviewer \# 3 - This work implemented the CRAFT block cipher with two architectures proposed, the Serial architecture and the Unrolled architecture. Then, the efficiency of the implementations is tested on three FPGA platforms. The implementations proposed in this paper are interesting for the practical applications of CRAFT in IoT.
	\end{quote}
\end{mdframed}

\subsection{Response}

We sincerely appreciate the time and effort you've invested in reviewing our paper. Your valuable suggestions are greatly appreciated and will undoubtedly enhance the quality of our work. In the subsequent section, we will address each of your comments individually.

\noindent\rule{\linewidth}{2.0pt}

\subsection{Reviewer Comment}
\begin{mdframed}
	\begin{quote}
		Reviewer \# 3.1 - The reference format should be consistent with equations, such as, 'as shown in Equation 1' $->$ 'as shown in Equation (1)'.
	\end{quote}
\end{mdframed}

\subsection{Response}

Thank you for your suggestion. We have revised the manuscript to ensure that the reference format is consistent with the equations.


\noindent\rule{\linewidth}{2.0pt}

\subsection{Reviewer Comment}
\begin{mdframed}
	\begin{quote}
		Reviewer \# 3.2 - The description of IA is not necessary to occupy a single subsection (Subsection 3.3). It is better to move it to Preliminaries. Meanwhile, the comparison between the IA, SA, and UA is not sufficient.
	\end{quote}
\end{mdframed}

\subsection{Response}

We appreciate your expert suggestions. We remove the subsection 3.3 IA and remove it to the Preliminaries (Section 2). We also add the discussion of the comparison between the IA, SA, and UA on the discussion section. The details are as follows:

\color{blue}
This section discusses the performance of three different architectures of CRAFT: the serial architecture, the unrolled architecture, and the iterative architecture. These architectures are compared and analyzed to determine which one is best suited for different environments.

The serial architecture of CRAFT is designed to minimize area consumption. It has the lowest area consumption among the three architectures, which results in it having the lowest dynamic power. Additionally, it boasts the highest frequency among the three architectures. However, it has a high latency, leading to the lowest maximum throughput among the three architectures. The serial architecture is suitable for resource-limited environments where high throughput is not a requirement.

The unrolled architecture of CRAFT aims to maximize throughput. It boasts the lowest latency among the three architectures, which contributes to its highest maximum throughput, despite having the lowest maximum frequency. This architecture is also energy-efficient, offering the lowest energy per bit among the three architectures. However, it does have the highest area consumption. The unrolled architecture is best suited for environments where high throughput and low energy are priorities, and low area is not a requirement.

The iterative architecture of CRAFT is designed to strike a balance between area consumption and throughput. While it doesn't have the lowest area consumption, the highest frequency, the lowest latency, the highest maximum throughput, or the lowest energy per bit among the three architectures, it does have the highest throughput per slice. The iterative architecture is suitable for environments where moderate throughput is required at the lowest possible area cost.
\color{black}

\noindent\rule{\linewidth}{2.0pt}

\subsection{Reviewer Comment}
\begin{mdframed}
	\begin{quote}
		Reviewer \# 3.3 - The definitions of SA and UA are not explicit, especially the implementation under UA. It is necessary to highlight their innovativeness in relation to previous architectures.
	\end{quote}
\end{mdframed}

\subsection{Response}

We appreciate your suggestion. To clarify, we have added the definitions of SA and UA in the headers of Subsections 3.1 and 3.2, respectively. Additionally, we have emphasized their innovative aspects compared to the architectures proposed in the original CRAFT article. The details are as follows:

\color{blue}
The purpose of the serial architecture is to reduce the datapath, which represents the number of bits dealt with in one cycle. For instance, the CRAFT cipher has a 64-bit block size, meaning it can process 64-bit data in one cycle when using the iterative architecture. In contrast, the serial architecture reduces the datapath from 64-bit to 4-bit, meaning it can only process 4-bit data in one cycle. Despite this limitation, serial architectures can significantly reduce area usage by reusing components, making them a viable alternative to iterative architectures.

The unrolled architecture allows for the execution of more than one round function in a single cycle. In contrast, the iterative architecture of the CRAFT cipher only runs one round function per cycle. This approach can significantly reduce the latency of the encryption process. It does this by reducing the number of cycles needed to encrypt one block size plaintext, thereby improving the throughput rate.
\color{black}

\noindent\rule{\linewidth}{2.0pt}

\subsection{Reviewer Comment}
\begin{mdframed}
	\begin{quote}
		Reviewer \# 3.4 - Can CRAFT be implemented under the Iterative architecture? If so, do implementations under UA or SA still have advantages over IA? On the other hand, can UA or SA be applied to other ciphers, such as PRESENT, and bring efficiency improvements?
	\end{quote}
\end{mdframed}

\subsection{Response}

todo for Response to Reviewer \# 3.4, it want a fair compare, give it a reason.

logic answer:

The CRAFT can be implemented under the Iterative architecture, but the UA and SA still have advantages over IA. The UA and SA can also be applied to other ciphers, such as PRESENT, to bring efficiency improvements. However, since different ciphers have different linear and non-linear operations, enabling the new components to work together poses a challenge under a common strategy.

\noindent\rule{\linewidth}{2.0pt}

\subsection{Reviewer Comment}
\begin{mdframed}
	\begin{quote}
		Reviewer \# 3.5 - It is better to combine the Section 4 with Section 5 in my opinion.
	\end{quote}
\end{mdframed}

\subsection{Response}

Thank you for your suggestion. To enhance the readability of the manuscript, we have merged Sections 4 and 5 into a new Section 4 titled "Experiment Results". The previous Section 5 has been transformed into Subsection 4.5.

\noindent\rule{\linewidth}{2.0pt}

\subsection{Reviewer Comment}
\begin{mdframed}
	\begin{quote}
		Reviewer \# 3.6 - The data in Figure 9-13 is also demonstrated in Table 5-7, one approach is enough to compare the results.
	\end{quote}
\end{mdframed}

\subsection{Response}

We appreciate your suggestion. While the data from Figures 9-13 is also presented in Tables 5-7, we believe that the graphical representation in the figures provides a more intuitive understanding. Therefore, we have decided to retain both the figures and the tables in our manuscript.

\noindent\rule{\linewidth}{6.0pt}
\section{Response to Reviewer \#1}
\subsection*{Overall Comments}
\begin{mdframed}
	\begin{quote}
		Reviewer \# 1 - Good paper. Talk briefly about side channel attacks. Address the comments below for another revision.
	\end{quote}
\end{mdframed}

\subsection{Response}

We greatly value your detailed feedback and careful review. In response to your suggestions, we have expanded our discussion on side channel attacks. Additionally, we have included relevant references to support this new content. We trust that these revisions satisfactorily address your concerns.

\noindent\rule{\linewidth}{2.0pt}

\subsection{Reviewer Comment}
\begin{mdframed}
	\begin{quote}
		Reviewer \# 1.1 - References are not uniformly formatted
	\end{quote}
\end{mdframed}

\subsection{Response}
We appreciate your feedback. We have meticulously reviewed the .bib file and the references, and made the necessary adjustments to ensure uniform formatting. We believe these revisions adequately address the concerns.

\noindent\rule{\linewidth}{2.0pt}

\subsection{Reviewer Comment}
\begin{mdframed}
	\begin{quote}
		Reviewer \# 1.2 - Please add comparisons in a table (or subsection) so that one could fairly compare your work with similar previous works.
	\end{quote}
\end{mdframed}

\subsection{Response}
% TODO add specific table, need add the table in the response
Thank you for your suggestion. In response, we have added a new table in the Results section of our revised manuscript. This table offers a comprehensive comparison between our work and similar previous studies. We trust that this addition effectively addresses your concerns. The detailed revisions are as follows:

\color{blue}

add compare table in here

\color{black}

\noindent\rule{\linewidth}{2.0pt}

\subsection{Reviewer Comment}
\begin{mdframed}
	\begin{quote}
		Reviewer \# 1.3 - Papers related to crypto need to consider this: With the advent of post-quantum cryptography, it is better to add some relevant papers including the followings to make sure you cover that topic too. When PQC replaces ECC/RSA every security application from smart phones to block chains will be affected.
	\end{quote}
\end{mdframed}

\subsection{Response}

We appreciate your feedback. As a result, we've addressed the significant topic of post-quantum cryptography in the Introduction section of our revised manuscript. To support this new content, we've also included pertinent references. We believe these modifications adequately address your concerns. The detailed revisions are as follows:


\color{blue}

To counter these fault attacks, various schemes have been proposed. For instance, [18] proposed a fault diagnosis scheme, while [19] introduced an error detection scheme. Additionally, [20] and [21] have suggested error detection methods specifically for Post-Quantum Cryptography.

	[18] M. M. Kermani, S. B. Sarmadi, A. E. Ackie, R. Azarderakhsh, High-performance fault diagnosis schemes for efficient hash algorithm BLAKE, in: R. S. Murphy (Ed.), 10th IEEE Latin American Symposium on Circuits \& Systems, LASCAS 2019, Armenia, Colombia, February 24-27, 2019, IEEE, 2019, pp. 201-204. doi:10.1109/LASCAS.2019.8667597.

[19] J. Kaur, A. C. Canto, M. M. Kermani, R. Azarderakhsh, Hardware constructions for error detection in wg-29 stream cipher benchmarked on fpga, IEEE Transactions on Computer-Aided Design of Integrated Circuits and Systems (2024) 1-1doi:10.1109/tcad.2023.3338108.

[20] A. Cintas-Canto, M. Mozaffari-Kermani, R. Azarderakhsh, K. Gaj, Crc-oriented error detection architectures of post-quantum cryptography niederreiter key generator on fpga (2022). doi:10.1109/norcas57515.2022.9934378.

[21] A. C. Canto, A. Sarker, J. Kaur, M. M. Kermani, R. Azarderakhsh, Error detection schemes assessed on fpga for multipliers in lattice-based key encapsulation mechanisms in post-quantum cryptography, IEEE Transactions on Emerging Topics in Computing 11 (3) (2023) 791-797. doi:10.1109/tetc.2022.3217006.

\color{black}

\noindent\rule{\linewidth}{2.0pt}

\subsection{Reviewer Comment}
\begin{mdframed}
	\begin{quote}
		Reviewer \# 1.4 - Again, any crypto paper needs to address this: Also add some previous works on side-channel attacks and lightweight cryptography or PQC.
	\end{quote}
\end{mdframed}

\subsection{Response}
% TODO add SCA content of Introduction

We appreciate your expert suggestion. In response, we have expanded our discussion to include previous works on side-channel attacks and have added relevant references to support this content. We trust that these modifications adequately address your concerns. The detailed revisions are as follows:

\color{blue}

Differential fault analysis, a type of side channel attack, was first proposed by [16] and further discussed in [17]. To counter these fault attacks, various schemes have been proposed. For instance, [18] proposed a fault diagnosis scheme, while [19] introduced an error detection scheme. Additionally, [20] and [21] have suggested error detection methods specifically for Post-Quantum Cryptography.

	[16] E. Biham, A. Shamir, Differential fault analysis of secret key cryptosystems, Springer Berlin Heidelberg, 1997, pp. 513-525. doi:10.1007/bfb0052259.


[17] M. M. Kermani, R. Azarderakhsh, M. Mirakhorli, Multidisciplinary approaches and challenges in integrating emerging medical devices security research and education (2016). \linebreak doi:10.18260/p.25761.


[18] M. M. Kermani, S. B. Sarmadi, A. E. Ackie, R. Azarderakhsh, High-performance fault diagnosis schemes for efficient hash algorithm BLAKE, in: R. S. Murphy (Ed.), 10th IEEE Latin American Symposium on Circuits \& Systems, LASCAS 2019, Armenia, Colombia, February 24-27, 2019, IEEE, 2019, pp. 201-204. doi:10.1109/LASCAS.2019.8667597.

[19] J. Kaur, A. C. Canto, M. M. Kermani, R. Azarderakhsh, Hardware constructions for error detection in wg-29 stream cipher benchmarked on fpga, IEEE Transactions on Computer-Aided Design of Integrated Circuits and Systems (2024) 1-1doi:10.1109/tcad.2023.3338108.

[20] A. Cintas-Canto, M. Mozaffari-Kermani, R. Azarderakhsh, K. Gaj, Crc-oriented error detection architectures of post-quantum cryptography niederreiter key generator on fpga (2022). doi:10.1109/norcas57515.2022.9934378.

[21] A. C. Canto, A. Sarker, J. Kaur, M. M. Kermani, R. Azarderakhsh, Error detection schemes assessed on fpga for multipliers in lattice-based key encapsulation mechanisms in post-quantum cryptography, IEEE Transactions on Emerging Topics in Computing 11 (3) (2023) 791-797. doi:10.1109/tetc.2022.3217006.


\color{black}

\noindent\rule{\linewidth}{2.0pt}

\subsection{Reviewer Comment}
\begin{mdframed}
	\begin{quote}
		Reviewer \# 1.5 - You could add a subsection for Discussions.
	\end{quote}
\end{mdframed}

\subsection{Response}
% TODO add a section of discussion

We value your expert suggestion. In response, we have added a Discussions section to improve the quality of our manuscript. The detailed revisions are as follows:

\color{blue}
section of discussion

\color{black}

\noindent\rule{\linewidth}{2.0pt}

\subsection{Reviewer Comment}
\begin{mdframed}
	\begin{quote}
		Reviewer \# 1.6 - Please add one or more future works for enhancing your presentation
	\end{quote}
\end{mdframed}

\subsection{Response}
% TODO in discussion section to talk future work

We appreciate your insightful suggestion. In response, we have included a discussion of potential future work in the Discussions section to enhance the quality of our manuscript. The detailed revisions are as follows:

\color{blue}

future work, section of discussion

\color{black}

\noindent\rule{\linewidth}{2.0pt}

\subsection{Reviewer Comment}
\begin{mdframed}
	\begin{quote}
		Reviewer \# 1.7 - Moreover, some works missing on lightweight cryptography LWC and building blocks.
	\end{quote}
\end{mdframed}

\subsection{Response}
% TODO add similar work on the introduction

We value your insightful suggestion. In response, we have included the previously missing implementation work and added the necessary references to support this addition. The detailed revisions are as follows:

\color{blue}

similar work on the introduction

	[xx] Optimized architectures for elliptic curve cryptography over Curve448, Cryptology ePrint Archive, 2020.

	[xx] Reliable architectures for finite field multipliers using cyclic codes on FPGA utilized in classic and post-quantum cryptography, IEEE Transactions on Very Large Scale Integration (VLSI) Systems 31 (1), 2022.

	[xx] Hyflex hands-on hardware security education during covid-19, 2022 IEEE World Engineering Education Conference (EDUNINE).
arXiv preprint arXiv:2306.08178

[xx] Introduction to the Special Section on Emerging Security Trends for Biomedical Computations, Devices, and Infrastructures, IEEE/ACM Transactions on Computational Biology and Bioinformatics, 2016.


\color{black}


\noindent\rule{\linewidth}{2.0pt}

\subsection{Reviewer Comment}
\begin{mdframed}
	\begin{quote}
		Reviewer \# 1.8 - With the advent of post-quantum cryptography (PQC), it is better to add some relevant works to make sure you cover that topic too. This is the hottest topic in cryptography now. With PQC, add a paper on each of these six topics separately: (a) Curve448 and Ed448 on Cortex-M4, (b) SIKE on Cortex-M4, (c) SIKE Round 3 on ARM Cortex-M4, (d) Kyber on 64-Bit ARM Cortex-A, (e) Cryptographic accelerators on Ed25519, (f) Supersingular isogeny Diffie-Hellman key exchange on 64-bit ARM.
	\end{quote}
\end{mdframed}

\subsection{Response}
% TODO add similar work on the introduction

We appreciate your expert suggestion. However, our paper's primary focus is on hardware implementation, which is why we did not include the suggested papers on software implementation. In response to your feedback, we have expanded the Introduction section of our revised manuscript. This expansion includes a discussion on post-quantum cryptography, backed by relevant references. We trust that these modifications adequately address your concerns. The detailed revisions are as follows:

\color{blue}
% TODO replace introduction with PQC discussion with similar the reviewer # 1.3

introduction with PQC discussion


	[xx] High-performance fault diagnosis schemes for efficient hash algorithm blake, 2019 IEEE 10th Latin American Symposium on Circuits \& Systems (LASCAS).

	[xx] CRC-oriented error detection architectures of post-quantum cryptography niederreiter key generator on FPGA, 2022 IEEE Nordic Circuits and Systems Conference (NorCAS), 2022.

	[xx] Error Detection Schemes Assessed in FPGA for Multipliers in Lattice-Based Key Encapsulation Mechanisms in post-quantum cryptography, IEEE Transactions on Emerging Topics in Computing, 2022.

	[xx] Hardware Constructions for Error Detection in WG-29 Stream Cipher Benchmarked on FPGA, IEEE Transactions on Computer-Aided Design of Integrated Circuits and Systems, 2023, to appear.

\color{black}

\noindent\rule{\linewidth}{2.0pt}

\subsection{Reviewer Comment}
\begin{mdframed}
	\begin{quote}
		Reviewer \# 1.9 - NIST lightweight standardization was finalized in Feb. 2023. Also mention fault attacks as side-channel attacks, these topics to explore and add a reference on each of these separately: (a) Error Detection in Lightweight Welch-Gong (WG)-Oriented Streamcipher WAGE, (b) error detection reliable architectures of Camellia block cipher, (c) fault diagnosis of low-energy Midori cipher, (d) block cipher QARMA with error detection mechanisms.
	\end{quote}
\end{mdframed}

\subsection{Response}

% TODO use the SCA content of Introduction
We appreciate your expert suggestion. In response, we have broadened our discussion in the Introduction section to include fault attacks as a type of side-channel attack. We've also added relevant references to support this new content. We believe these modifications adequately address your concerns. The detailed revisions are as follows:

\color{blue}

detailed content of SCA in Introduction



\color{black}


\noindent\rule{\linewidth}{6.0pt}
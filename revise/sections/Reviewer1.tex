\section{Response to Reviewer \#1}
\subsection*{Overall Comments}
\begin{mdframed}
	\begin{quote}
		Reviewer \# 1 - Good paper. Talk briefly about side channel attacks. Address the comments below for another revision.
	\end{quote}
\end{mdframed}

\subsection{Response}

We greatly value your detailed feedback and careful review. In response to your suggestions, we have expanded our discussion on side channel attacks. Additionally, we have included relevant references to support this new content. We trust that these revisions satisfactorily address your concerns.

\noindent\rule{\linewidth}{2.0pt}

\subsection{Reviewer Comment}
\begin{mdframed}
	\begin{quote}
		Reviewer \# 1.1 - References are not uniformly formatted
	\end{quote}
\end{mdframed}

\subsection{Response}
We appreciate your feedback. We have meticulously reviewed the .bib file and the references, and made the necessary adjustments to ensure uniform formatting. We believe these revisions adequately address the concerns.

\noindent\rule{\linewidth}{2.0pt}

\subsection{Reviewer Comment}
\begin{mdframed}
	\begin{quote}
		Reviewer \# 1.2 - Please add comparisons in a table (or subsection) so that one could fairly compare your work with similar previous works.
	\end{quote}
\end{mdframed}

\subsection{Response}

We appreciate your suggestion. In response, we've conducted a fair comparison with similar previous works, specifically the Iterative architecture of CRAFT proposed in the original paper. We've updated our manuscript to include a description of the three architectures in Table 4, and a comparison in Tables 5-7. We believe these updates ensure a fair comparison of our work with similar previous studies. Thank you once again for your valuable suggestion.



\noindent\rule{\linewidth}{2.0pt}

\subsection{Reviewer Comment}
\begin{mdframed}
	\begin{quote}
		Reviewer \# 1.3 - Papers related to crypto need to consider this: With the advent of post-quantum cryptography, it is better to add some relevant papers including the followings to make sure you cover that topic too. When PQC replaces ECC/RSA every security application from smart phones to block chains will be affected.
	\end{quote}
\end{mdframed}

\subsection{Response}

We appreciate your feedback. As a result, we've addressed the significant topic of post-quantum cryptography in the Introduction section of our revised manuscript. To support this new content, we've also included pertinent references. We believe these modifications adequately address your concerns. The detailed revisions are as follows:


\color{blue}

In response to these fault attacks, several countermeasures have been suggested. For example, [18] introduced a scheme for error detection. In the realm of Post-Quantum Cryptography, [19] proposed specific error detection methods.

	[18] J. Kaur, A. C. Canto, M. M. Kermani, R. Azarderakhsh, Hardware constructions for error detection in wg-29 stream cipher benchmarked on fpga, IEEE Transactions on Computer-Aided Design of Integrated Circuits and Systems (2024) 1-1doi:10.1109/tcad.2023.3338108.

[19] A. C. Canto, A. Sarker, J. Kaur, M. M. Kermani, R. Azarderakhsh, Error detection schemes assessed on fpga for multipliers in lattice-based key encapsulation mechanisms in post-quantum cryptography, IEEE Transactions on Emerging Topics in Computing 11 (3) (2023) 791-797. doi:10.1109/tetc.2022.3217006.

\color{black}

\noindent\rule{\linewidth}{2.0pt}

\subsection{Reviewer Comment}
\begin{mdframed}
	\begin{quote}
		Reviewer \# 1.4 - Again, any crypto paper needs to address this: Also add some previous works on side-channel attacks and lightweight cryptography or PQC.
	\end{quote}
\end{mdframed}

\subsection{Response}
% TODO add SCA content of Introduction

We appreciate your expert suggestion. In response, we have expanded our discussion to include previous works on side-channel attacks and have added relevant references to support this content. We trust that these modifications adequately address your concerns. The detailed revisions are as follows:

\color{blue}

Differential fault analysis, which is a type of side channel attack, was first introduced by [16]. This concept was later elaborated in more detail by [17]. In response to these fault attacks, several countermeasures have been suggested. For example, [18] introduced a scheme for error detection. In the realm of Post-Quantum Cryptography, [19] proposed specific error detection methods.

	[16] E. Biham, A. Shamir, Differential fault analysis of secret key cryptosystems, Springer Berlin Heidelberg, 1997, pp. 513-525. doi:10.1007/bfb0052259.


[17] M. M. Kermani, R. Azarderakhsh, M. Mirakhorli, Multidisciplinary approaches and challenges in integrating emerging medical devices security research and education (2016). \linebreak doi:10.18260/p.25761.


[18] J. Kaur, A. C. Canto, M. M. Kermani, R. Azarderakhsh, Hardware constructions for error detection in wg-29 stream cipher benchmarked on fpga, IEEE Transactions on Computer-Aided Design of Integrated Circuits and Systems (2024) 1-1doi:10.1109/tcad.2023.3338108.

[19] A. C. Canto, A. Sarker, J. Kaur, M. M. Kermani, R. Azarderakhsh, Error detection schemes assessed on fpga for multipliers in lattice-based key encapsulation mechanisms in post-quantum cryptography, IEEE Transactions on Emerging Topics in Computing 11 (3) (2023) 791-797. doi:10.1109/tetc.2022.3217006.


\color{black}

\noindent\rule{\linewidth}{2.0pt}

\subsection{Reviewer Comment}
\begin{mdframed}
	\begin{quote}
		Reviewer \# 1.5 - You could add a subsection for Discussions.
	\end{quote}
\end{mdframed}

\subsection{Response}

We value your expert suggestion. In response, we have added a Discussions section to improve the quality of our manuscript. The detailed revisions are as follows:

\color{blue}

This section discusses the performance of three different architectures of CRAFT: the serial architecture, the unrolled architecture, and the iterative architecture. These architectures are compared and analyzed to determine which one is best suited for different environments.

The serial architecture of CRAFT is designed to minimize area consumption. It has the lowest area consumption among the three architectures, which results in it having the lowest dynamic power. Additionally, it boasts the highest frequency among the three architectures. However, it has a high latency, leading to the lowest maximum throughput among the three architectures. The serial architecture is suitable for resource-limited environments where high throughput is not a requirement.

The unrolled architecture of CRAFT aims to maximize throughput. It boasts the lowest latency among the three architectures, which contributes to its highest maximum throughput, despite having the lowest maximum frequency. This architecture is also energy-efficient, offering the lowest energy per bit among the three architectures. However, it does have the highest area consumption. The unrolled architecture is best suited for environments where high throughput and low energy are priorities, and low area is not a requirement.

The iterative architecture of CRAFT is designed to strike a balance between area consumption and throughput. While it doesn't have the lowest area consumption, the highest frequency, the lowest latency, the highest maximum throughput, or the lowest energy per bit among the three architectures, it does have the highest throughput per slice. The iterative architecture is suitable for environments where moderate throughput is required at the lowest possible area cost.

\color{black}

\noindent\rule{\linewidth}{2.0pt}

\subsection{Reviewer Comment}
\begin{mdframed}
	\begin{quote}
		Reviewer \# 1.6 - Please add one or more future works for enhancing your presentation
	\end{quote}
\end{mdframed}

\subsection{Response}

We appreciate your insightful suggestion. In response, we have included a discussion of potential future work in the Conclusion section to enhance the quality of our manuscript. The detailed revisions are as follows:

\color{blue}

Future work could extend the proposed architectures to other lightweight ciphers and examine their performance. Additionally, these lightweight ciphers could be implemented in a way that makes them resistant to side channel attacks.

\color{black}

\noindent\rule{\linewidth}{2.0pt}

\subsection{Reviewer Comment}
\begin{mdframed}
	\begin{quote}
		Reviewer \# 1.7 - Moreover, some works missing on lightweight cryptography LWC and building blocks.
	\end{quote}
\end{mdframed}

\subsection{Response}

We value your insightful suggestion. The similar works we have talk in the introduction, and we have added relevant references to support this content. The detailed contents are as follows:

\color{blue}

Lara-Nino et al. [20] introduced a 16-bit datapath architecture for the PRESENT cipher, which resulted in reduced area and power consumption. Similarly, Pandey et al. [21] suggested an optimized key schedule for the same cipher, leading to a smaller area. Shahbazi et al. [22] put forth an 8-bit serial architecture for AES, which also reduces area and power consumption. Li et al. [23] presented unrolled architectures and a low-cost architecture for PRINCE, optimizing both throughput and area separately.

	[20] C. A. Lara-Nino, A. Diaz-Perez, M. Morales-Sandoval, Lightweight hardware architectures for the present cipher in FPGA, IEEE Trans. Circuits Syst. I Regul. Pap. 64-I (9) (2017) 2544-2555. doi:10.1109/TCSI.2017.2686783.

[21] J. G. Pandey, T. Goel, A. Karmakar, Hardware architectures for PRESENT block cipher and their FPGA implementations, IET Circuits Devices Syst. 13 (7) (2019) 958-969. doi:10.1049/IET-CDS.2018.5273.

[22] K. Shahbazi, S. Ko, Area-efficient nano-aes implementation for internet-of-things devices, IEEE Trans. Very Large Scale Integr. Syst. 29 (1) (2021) 136-148. doi:10.1109/TVLSI.2020.3033928.

[23] L. Li, J. Feng, B. Liu, Y. Guo, Q. Li, Implementation of PRINCE with resource-efficient structures based on fpgas, Frontiers Inf. Technol. Electron. Eng. 22 (11) (2021) 1505–1516. doi:10.1631/FITEE.2000688.

\color{black}

\noindent\rule{\linewidth}{2.0pt}

\subsection{Reviewer Comment}
\begin{mdframed}
	\begin{quote}
		Reviewer \# 1.8 - With the advent of post-quantum cryptography (PQC), it is better to add some relevant works to make sure you cover that topic too. This is the hottest topic in cryptography now. With PQC, add a paper on each of these six topics separately: (a) Curve448 and Ed448 on Cortex-M4, (b) SIKE on Cortex-M4, (c) SIKE Round 3 on ARM Cortex-M4, (d) Kyber on 64-Bit ARM Cortex-A, (e) Cryptographic accelerators on Ed25519, (f) Supersingular isogeny Diffie-Hellman key exchange on 64-bit ARM.
	\end{quote}
\end{mdframed}

\subsection{Response}


We appreciate your expert suggestion. However, our paper's primary focus is on hardware implementation, which is why we did not include the suggested papers on software implementation. In response to your feedback, we have expanded the Introduction section of our revised manuscript. This expansion includes a discussion on post-quantum cryptography, backed by relevant references. We trust that these modifications adequately address your concerns. The detailed revisions are as follows:

\color{blue}

In response to these fault attacks, several countermeasures have been suggested. For example, [18] introduced a scheme for error detection. In the realm of Post-Quantum Cryptography, [19] proposed specific error detection methods.

	[18] J. Kaur, A. C. Canto, M. M. Kermani, R. Azarderakhsh, Hardware constructions for error detection in wg-29 stream cipher benchmarked on fpga, IEEE Transactions on Computer-Aided Design of Integrated Circuits and Systems (2024) 1-1doi:10.1109/tcad.2023.3338108.

[19] A. C. Canto, A. Sarker, J. Kaur, M. M. Kermani, R. Azarderakhsh, Error detection schemes assessed on fpga for multipliers in lattice-based key encapsulation mechanisms in post-quantum cryptography, IEEE Transactions on Emerging Topics in Computing 11 (3) (2023) 791-797. doi:10.1109/tetc.2022.3217006.

\color{black}

\noindent\rule{\linewidth}{2.0pt}

\subsection{Reviewer Comment}
\begin{mdframed}
	\begin{quote}
		Reviewer \# 1.9 - NIST lightweight standardization was finalized in Feb. 2023. Also mention fault attacks as side-channel attacks, these topics to explore and add a reference on each of these separately: (a) Error Detection in Lightweight Welch-Gong (WG)-Oriented Streamcipher WAGE, (b) error detection reliable architectures of Camellia block cipher, (c) fault diagnosis of low-energy Midori cipher, (d) block cipher QARMA with error detection mechanisms.
	\end{quote}
\end{mdframed}

\subsection{Response}

We appreciate your expert suggestion. In response, we have broadened our discussion in the Introduction section to include fault attacks as a type of side-channel attack. We've also added relevant references to support this new content. We believe these modifications adequately address your concerns. The detailed revisions are as follows:

\color{blue}

Differential fault analysis, which is a type of side channel attack, was first introduced by [16]. This concept was later elaborated in more detail by [17]. In response to these fault attacks, several countermeasures have been suggested. For example, [18] introduced a scheme for error detection. In the realm of Post-Quantum Cryptography, [19] proposed specific error detection methods. It's important to note that these methods do increase the hardware consumption of the cipher system. The CRAFT cipher was designed with resistance to fault attacks in mind. However, it currently lacks efficient implementations. To make it suitable for use in more constrained environments, development of more efficient implementations is necessary.

	[16] E. Biham, A. Shamir, Differential fault analysis of secret key cryptosystems, Springer Berlin Heidelberg, 1997, pp. 513-525. doi:10.1007/bfb0052259.


[17] M. M. Kermani, R. Azarderakhsh, M. Mirakhorli, Multidisciplinary approaches and challenges in integrating emerging medical devices security research and education (2016). \linebreak doi:10.18260/p.25761.


[18] J. Kaur, A. C. Canto, M. M. Kermani, R. Azarderakhsh, Hardware constructions for error detection in wg-29 stream cipher benchmarked on fpga, IEEE Transactions on Computer-Aided Design of Integrated Circuits and Systems (2024) 1-1doi:10.1109/tcad.2023.3338108.

[19] A. C. Canto, A. Sarker, J. Kaur, M. M. Kermani, R. Azarderakhsh, Error detection schemes assessed on fpga for multipliers in lattice-based key encapsulation mechanisms in post-quantum cryptography, IEEE Transactions on Emerging Topics in Computing 11 (3) (2023) 791-797. doi:10.1109/tetc.2022.3217006.


\color{black}


% \noindent\rule{\linewidth}{6.0pt}